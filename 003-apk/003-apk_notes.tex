%<*Composition>
\textbf{\uline{AndroidManifest}}\\
\textbullet~~Fichier xml qui contient:\\
\textbullet~~Permissions\\
\textbullet~~Liste des activités\\
\textbullet~~Services
\smallbreak
\textbf{\uline{Assets}}\\
Autres resources (polices...)
\smallbreak
\textbf{\uline{Classes.dex}}\\
Code binaire de l'application, limité à 65 000 méthodespar fichier,
mais il faut éviter de le dépasser, parce que le support au-delas n'est pas assuré
\smallbreak

\textbf{\uline{META-INF}}
\smallbreak
Informations autour de l'application:\\
\textbullet~~La classe à lancer\\
\textbullet~~ Version du package\\
\textbullet~~ Numéro de version\\
\textbf{\uline{res}}\\
Ressources non compilées, mais standares\\
ex: Layout, Drawables, 
\smallbreak
\textbf{\uline{resources.arsc}}\\
Resources compilées, en format binaires xml
%</Composition>

%<*Compilation>
Le code java est compilé en byte-code Java, puis rassemblé,
compacté et optimisé via le programme dx en un executable dalvik,
byte-code spécifique, prêt à être exploité par la machine
virtuelle dalvik ou ART.
\bigbreak
Ce fichier n'est pas compressé, ce qui permet à la machine virtuelle de mapper
le code en mémoire et de le partager très facilement\\
%</Compilation>

%<*ARTVSDalvik>
\textbf{\uline{JIT:}}\\
\textbullet~~ Just In time:\\
L'application est compilé en langage machine uniquement au moment 
ou le code est nécessaire
Occupe moins d'expace mémoire\\
Plus lents\\
Disponible pour les versions inférieurs à Kitkat\\
\textbf{\uline{AOT:}}\\
\textbullet~~ Ahead of time:\\
L'application est compilé en langage machine au moment de son installation
Bien plus rapide\\
Occupe plus d'expace mémoire\\
Disponible pour les versions supérieures à Kitkat
\\
A partir de nougat, ART a été modifié pour utiliser à la fois un comportement JIT et AOT, 
permettant d'allier le meilleur des deux méthodes
%</ARTVSDalvik>
