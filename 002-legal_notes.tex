%<*propIntelect>
Si le logiciel peut être protégé par une licence, sa fonctionnalité en tant que telle n’est pas protégeable.
\smallbreak
Par exemple, le programme Word inclus dans la suite bureautique: 
\smallbreak
  \textbullet~~Office de Windows fait l’objet d’une protection juridique spécifique.
  \smallbreak
  \textbullet~~Mais la fonctionnalité de Word, à savoir être un logiciel de
  traitement de texte, n’est pas protégeable par Microsoft.
%</propIntelect>

%<*loi>
\bigbreak
Le logiciel ne doit pas avoir été piraté
\smallbreak
Et :\smallbreak
\textbullet~~Soit la license le permet
\smallbreak
\textbullet~~Soit cette rétroingénierie est effectué dans un but d'interropérabilité,
 %</loi>

 %<*interropérabilité>
 \bigbreak
 L’interopérabilité est la faculté pour un logiciel à fonctionner sur l’ensemble des
 plateformes existantes.
 %</interropérabilité>

 %<*Resume>
\bigbreak
En résumé, il est interdit de décompiler une application à des fins autres 
 que de rétroingénierie, ce qui fait que pour analyser une application, il est 
illégale d'analyser le code source, et encore plus de le partager
 %</Resume>

 %<*BugBounty>
 \textbf{\uline{Bug Bounty :}}
Un bug Bounty est une récompense qu'une société offre à  tous ceux qui trouvent des failles
 de sécurité dans un périmètre donné.\\
Ce périmètre peut être un site web, une application, une API...etc., etc. C'est donc à
 l'entreprise de déterminer les services que les gens vont pouvoir explorer à  la recherche
  de failles de sécurité.\\
Il y a bien sûr des règles à  respecter et chaque Bug Bounty doit énoncer clairement les
 limites que le hacker ou l'expert ne doit pas franchir, mais en général, comme ça se passe
  sur des services en production, il vaut mieux éviter de tout casser si on veut sa récompense.\\

D'ailleurs, concernant le montant de la récompense, c'est assez variable d'une société à
 l'autre et ça dépend surtout du type de faille remontée. Plus la faille est critique,
  complexe, bien documentée avec si possible un PoC (Proof of concept) et pourquoi pas
   des recommandations, voire un patch, plus la récompense sera grande.

Évidemment, si vous trouvez des failles qui ont déjà  été trouvées par un autre,
 vous ne recevrez aucune récompense.
 \bigbreak
%</BugBounty>

%<*Tweet>
Cela permet également d'éviter, pour les entreprises ce genre de choses.\\
Etant donné qu'apple ne fournit pas de bugbounty pour macOS, un chercheur a révélé publiquement
 une faille critique sur macOS.\\
 Cependant, il précise qu'il a pris cette décision car cette vulnérabilité n'est pas accessible 
  à distance
%</Tweet>












L’ingénierie inverse, rétro-conception ou « reverse engineering » est l’« activité qui consiste à étudier
un objet pour en déterminer le fonctionnement interne ou la méthode de fabrication » 45 . Ainsi, grâce
aux méthodes de rétro-conception, il est possible de comprendre le comportement du logiciel, sans
en détenir ni le code source ni la documentation 6 .

Si les éditeurs sont nombreux à interdire ou limiter l’ingénierie inversée de leurs logiciels dans leurs
conditions d’utilisation, l’ingénierie inverse reste une pratique légale dans de nombreux pays, bien
qu’encadrée. Elle est notamment autorisée à des fins de d’interopérabilité. Et dans les pays où cette
pratique est autorisée, les clauses ne sont pas valables, ou le sont dans les limites définies par la loi.
Enfin, l’ingénierie inversée est gouvernée par le droit de la propriété intellectuelle, droit propre à
chaque Etat. Il en découle une relative insécurité juridique pour les prestataires de services ou les
chercheurs.


C’est l’article L 122-6-1 du code de la propriété intellectuelle français qui autorise le reverse
engineering. Mais cette autorisation est strictement encadrée.
Le principe est celui du droit de propriété exclusif et opposable de l’auteur du logiciel, droit concédé
partiellement aux « clients » ou « acheteurs » par le biais de licences d’utilisation, par exemple.
L’article 122-6-1 qui autorise l’ingénierie inversée constitue une exception à ce système. Exception
envisagée dans deux cas de figure : le premier est l’exigence d’interopérabilité, le second l’utilisation
conforme à la destination du logiciel.


Le reverse engineering d’un logiciel à des fins d’interopérabilité permet, grâce à une technique de
décompilation, d’autoriser le logiciel à « échanger des informations » et « utiliser mutuellement des
informations » avec d’autres logiciels. 22 Souvent indispensable pour la création d’outil compatibles
avec certains logiciels ou terminaux, le reverse engineering reste encadré à des fins de protection de
la propriété intellectuelle. L’objectif étant d’exclure tout espionnage industriel, copie, ou
contournement de mesures de sécurité. Pour ce faire, trois conditions sont à respecter :



Que les informations permettant l’interopérabilité ne soient accessibles par aucun autre
moyen ;
Que le reverse engineering ne soit exercé que par l’utilisateur légitime du logiciel ;
Que le reverse engineering ne concerne que les parties du logiciel directement concernées
par l’interopérabilité.
C’est le contrat qui définira la finalité, la « destination » du logiciel. Ce système est également celui
proposé par le droit finlandais.

1.1.1.2.2 Le contournement des mesures de protection
Le reverse engineering ne doit pas « porter atteinte sciemment, à des fins autres que la recherche
[aux mesures de protection du droit d’auteur] » 23 .


« On n'a donc pas le droit en France de démontrer techniquement qu'un logiciel présente des failles
de sécurité, ou que la publicité pour ces logiciels est mensongère. Dormez tranquilles, citoyens, tous
vos logiciels sont parfaits. » 28
Cette citation du chercheur agissant sous le pseudonyme de Guillermito fait suite à sa condamnation
pour avoir « reproduit, modifié, et rassemblé tout ou partie du logiciel Viguard puis procédé à la
distribution gratuite de logiciels tirés des sources du logiciel Viguard » 29 , en contradiction avec la
licence. « Le fait que celui qui agissait sous le pseudonyme Guillermito ait commis ces actes dans le
but de détecter les failles de sécurité du logiciel Viguard et d’en faire profiter la communauté dans
les forums de discussion n’entre pas dans le cadre de [l’exception] prévue au monopole de l’auteur
[prévue par l’article L122-6-1 du Code de procédure pénale]. » 30

En droit français, la simple recherche de failles de sécurité n’est pas réprimée en tant que telle, sous
réserve de respect du régime juridique relatif au respect du droit de la propriété intellectuelle. C’est la publication des résultats, la « démonstration technique » de ces recherches qui peut être
sanctionnée par l’article 323-3-1 du Code pénal.


Full et responsible disclosure 32
Il existe plusieurs chapelles parmi les passionnés de sécurité informatique. Certains vont informer le
propriétaire du site internet, ou l’éditeur du logiciel, de l’existence de cette faille afin que ce dernier
la corrige au plus vite : c’est ce qu’on appelle un « white hat » 33 . Parmi ceux-ci, certains sont
partisans du « full disclosure » ou divulgation complète, alors que d’autres prônent le « responsible
disclosure » ou divulgation responsable.
La différence entre les deux positions tient à l’étendue des informations révélées. Dans le cas de la
divulgation « complète », toutes les informations connues concernant la faille sont publiées, y
compris les « exploits », c’est-à-dire les moyens d’exploiter la faille. L’idée est que la faille sera plus
rapidement prise au sérieux et corrigée si toutes les données à son sujet sont rendues publiques. Les
partisans de la divulgation « responsable » choisissent de laisser un certain temps à l’intéressé pour
corriger la faille, et s’ils choisissent de divulguer l’existence d’une faille, ils ne fournissent en principe
pas les « exploits ». D’autres enfin estiment qu’effectuer une divulgation complète mais dans un
cercle d’initiés restreints constitue également une « responsible disclosure ».

L’article 323-3-1 34 du Code pénal sanctionne « le fait, sans motif légitime, d'importer, de détenir,
d'offrir, de céder ou de mettre à disposition un équipement, un instrument, un programme
informatique ou toute donnée conçus ou spécialement adaptés pour commettre une ou plusieurs des
infractions prévues par les articles 323-1 à 323-3 », c’est-à-dire le fait d'accéder, de se maintenir
frauduleusement, d’entraver, de fausser le fonctionnement ou d’introduire des données dans un
systèmes de traitement automatisé de données [STAD]. Ici, c’est le fait d’offrir, de céder ou de
mettre à disposition qui remet en cause ces pratiques de « disclosure ».


Il est légitime d’envisager que la finalité de sécurité informatique du reverse engineering entre dans
la définition du motif légitime évoqué dans l’article 323-3-1 du Code pénal. Mais ce constat amène
plusieurs interrogations : Ne peut-on pas imaginer que le droit à l’information constitue un motif
légitime de possession ou de diffusion d’un des outils incriminés par l’article 323-3-1 du Code pénal ?
D’autre part, doit-on apprécier la « finalité de sécurité informatique » de façon directe ou indirecte ?
En effet, le motif légitime ne manquera pas d’être invoqué de façon indirecte par certains « white
hat » pour justifier des « full disclosure ». Leur raisonnement sera de justifier des divulgations
complètes en expliquant que ces dernières placent les titulaires des droits sur les sites ou les logicielsvulnérables dos au mur, et les oblige à intervenir car la faille est accessible à n’importe quel « script
kiddies » 35 .


1.2 Chapitre 2. Le cas particulier du reverse engineering de malwares


Bien évidemment, le malware n’est pas protégé 37 par la propriété intellectuelle. Son analyse ne peut
donc être soumise au régime juridique décrit ci-dessus. En tant que tel, l’ingénierie inversée de
malware est légale. Mais plus que le fait de procéder à une analyse du malware, c’est sa possession
qui risque de tomber sous le coup de l’article 323-3-1 du Code pénal.

L’article 323-3-1 39 du Code pénal sanctionne « le fait, sans motif légitime, d'importer, de détenir,
d'offrir, de céder ou de mettre à disposition un équipement, un instrument, un programme
informatique ou toute donnée conçus ou spécialement adaptés pour commettre une ou plusieurs
des infractions prévues par les articles 323-1 à 323-3 », c’est-à-dire le fait d'accéder, de se maintenir
frauduleusement, d’entraver, de fausser le fonctionnement ou d’introduire des données dans un
systèmes de traitement automatisé de données [STAD]. Ce comportement « est puni des peines
prévues respectivement pour l'infraction elle-même ou pour l'infraction la plus sévèrement
réprimée », c’est-à-dire au maximum 7 ans de prison et 100 000€ d’amende.


