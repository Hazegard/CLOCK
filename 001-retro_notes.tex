%<*Definition>
La rétro-ingénierie logicielle est le principe d'analyser un programme sans ses sources, pour en comprendre le fonctionnement interne.
Deux approches complémentaires sont en général utilisées, l'analyse statique et l'analyse dynamique.
Dans le premier cas il s'agit de reconstituer le code source du logiciel à partir d'un exécutable ou au moins de le traduire dans le langage assembleur.
Pour le second cas, il s'agit d'étudier le programme directement pendant son exécution à l'aide d'un débogueur. 
%</Definition>

%<*Objectifs>
\smallbreak
\textbf{\uline{Interopérabilité d'un logiciel:}}\\ afin d’en comprendre le fonctionnement et ainsi le rendre compatible avec d'autres logiciels
\smallbreak
\textbf{\uline{documentation:}}\\ Retrouver le fonctionnement d’un logiciel avec lequelle on souhaiterait communiquer,
mais dont la documentation n’est plus disponible. 
\smallbreak
\textbf{\uline{veille compétitive:}}\\ Étudier mes produits concurrents, les méthodes utilisées, estimer les couts de développement d’une application similaire.
Mais cela permet également de déceler d’éventuelles violations de brevet par un concurrent. 
\smallbreak
\textbf{\uline{recherche de failles de sécurité:}}\\ C’est ainsi que certaines failles de sécurités sont trouvées dans les applications commerciales dont les sources ne sont pas disponibles. Les Virus sont eux aussi systématiquement étudiés par rétro ingénierie. D'ailleurs, ils sont très souvent très bien protégés afin de rendre leur identification plus lente. 
\smallbreak
\textbf{\uline{Piratage:}}\\ Prolonger la période d'essaie. `amtlib.dll' par exemple (adobe). 

%</Objectifs>



a la fois par rapprt
→ au seins

au cours de mon cursus je me suis orienté vers le dev car cela moffrait l'oopotunité de découvrir.

Par exemple, 

Mais m'a aussi permis d'acquéir
(E-businesss et mobilité)
tels


acquerir dans ce domaine.
J'ai par exemple aprris à réalise rdes applis web en react, mais je me suis également intéressé aux diffé tech pour sécuriser

Par ailleur

Date au danemark

Reformuler les concours, nominaliser