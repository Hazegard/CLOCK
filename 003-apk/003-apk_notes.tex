%<*Composition>
\textbf{\uline{Composition:}}
\smallbreak
\uline{AndroidManifest}\\
\textbullet~~Fichier xml qui contient:\\
\textbullet~~Permissions\\
\textbullet~~Liste des activités\\
\textbullet~~Services
\smallbreak
\uline{Assets}\\
Autres resources (polices...)
\smallbreak
\uline{Classes.dex}\\
Code binaire de l'application, limité à 65 000 méthodes par fichier
\smallbreak
\uline{META-INF}
\smallbreak
Informations autour de l'application:\\
\textbullet~~La classe à lancer\\
\textbullet~~Version du package\\
\textbullet~~Numéro de version\\
\uline{res}\\
Ressources non compilées, mais standares\\
ex: Layout, Drawables, 
\smallbreak
\uline{resources.arsc}\\
Resources compilées, en format binaires xml
%</Composition>

%<*Compilation>
\textbf{\uline{Compilation:}}\smallbreak
\textbullet~~Code java compilé en byte-code Java\\
\textbullet~~Rassemblé, compacté et optimisé via le programme dx en un executable dalvik,
~byte-code spécifique, prêt à être exploité par la machine virtuelle dalvik ou ART.
\bigbreak
Fichier non compressé: la machine virtuelle peut mapper rapidement
le code en mémoire et de le partager très facilement\\
%</Compilation>

%<*ARTVSDalvik>
\bigbreak
\textbf{\uline{JIT:}}\\
\textbullet~~ Just In time:\\
L'application est compilé en langage machine uniquement au moment~
où le code est nécessaire\\
\textbullet~~Occupe moins d'expace mémoire\\
\textbullet~~Plus lent\\
\textbullet~~Disponible pour les versions inférieurs à Kitkat\\
\textbf{\uline{AOT:}}\\
\textbullet~~ Ahead of time:\\
L'application est compilé en langage machine au moment de son installation\\
\textbullet~~Bien plus rapide\\
\textbullet~~Occupe plus d'expace mémoire\\
\textbullet~~Disponible pour les versions supérieures à Kitkat
\\
A partir de nougat, ART a été modifié pour utiliser à la fois un comportement JIT et AOT, 
~permettant d'allier le meilleur des deux méthodes
%</ARTVSDalvik>
