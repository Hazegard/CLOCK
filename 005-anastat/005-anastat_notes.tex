%<*Definition>
\textbf{\uline{L'analyse statique:}}\\
Analyse du logiciel réalisée sans exécuter le programme.\\
\uline{But:}trouver les défauts présents dans le logiciel.\\
S´assurer que le code est écrit selon des régles de progammations définis,
\bigbreak
%</Definition>

%<*Outils>
\textbf{\uline{Outils:}}\\
\textbullet~~Jadx: décompileur dex vers Java: ne pas avoir à manipuler 
 les fichiers smali (=assembleur pour android), qui sont peu lisibles\\
\textbullet~~Exodus-standalone: Analyse par signature pour déterminer les permissions
ainsi que les trackers utilisés par une application 
\bigbreak
   %</Outils>

%<*Objectifs>
\textbf{\uline{Objectifs:}}\\
\textbullet~~Permissions: déterminer si l'application nécessite des permissions 
 qui sembles incohérentes\\
 \textbullet~~Lister les trackers qui y sont inclus\\
 \textbullet~~Déterminer si des portions de code peuvent être intéressantes pour 
  l'analyse dynamique
\bigbreak
%</Objectifs>

%<*Methodes>
\textbf{\uline{Méthodes:}}\\
\textbullet~~Analyse du code source: peut être illégale\\
\textbullet~~Analyse par signature: pas de décompilation, ce qui rend l'opération un peu plus légale\\
 (mais pas forcément approuvée par les créateurs des applications que l'on analyse)
\bigbreak
%</Methodes>

%<*Exemple>
\textbf{\uline{sendPhoto:}} Méthode qui envoie des photos à un serveur distant\\
\textbf{\uline{sendSMS:}}Méthode qui envoie un SMS
%</Exemple>