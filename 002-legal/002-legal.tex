\documentclass[aspectratio=1610]{beamer}%, handout
  \usepackage{graphics}
\usepackage{pifont}
\usepackage{ulem}
\usepackage{xcolor}
\usepackage{modifycolor}
\usepackage{lipsum}
\usepackage{fontspec}
\usepackage{caption}
\renewcommand{\figurename}{FIGURE}
\captionsetup[figure]{labelfont={color=leftFootlineColor, scriptsize}, textfont={color=normalBlockColor, scriptsize}}
\usepackage{tabularx}
\newcolumntype{Y}{>{\centering\arraybackslash}X}
\usepackage{tikz}
\usepackage{standalone}
\usepackage{svg}
\usepackage{multicol}
\usepackage{catchfilebetweentags}
\usepackage{xifthen}
\usetikzlibrary{arrows}
\usetikzlibrary{backgrounds}
\setmonofont[
  Contextuals={Alternate}
]{FuraCode Nerd Font}
%\usepackage[T1]{fontspec}
%\usepackage[scaled]{berasans}
%\usepackage[scale=1,sfdefault,light]{roboto}
\setsansfont{Roboto Medium}
\usepackage{pgfpages}


\usepackage[cache=false,outputdir=build]{minted}
\definecolor{bg_code}{HTML}{282828}
\usemintedstyle{darcula}
\setminted{
      fontsize=\scriptsize, 
      linenos,
      numbersep=0pt,
      gobble=5,
      framesep=3mm} 
      \renewcommand{\theFancyVerbLine}{\texttt{{{\arabic{FancyVerbLine}}}}}
\usepackage{adjustbox}
\usepackage{environ}% Required for \NewEnviron, i.e. to read the whole body of the environment
%blue/red/pink/purple/yellow/orange/green/gray/bluegray/brown
\usepackage{tikz}
\usetikzlibrary{patterns}
\usepackage[french]{babel}
% This is the file main.tex
\usepackage{beamerthemesideblue}


\setbeamerfont{note page}{size=\footnotesize}
\setbeamertemplate{note page}[custom]
\setbeamercolor{note page}{bg=backgroundColor, fg=white}

%\usecolortheme{red}
\logo{\includegraphics[height=0.70cm]{logo.png}}
\title[]{La rétroingénierie appliquée à Android}
\subtitle{La traque aux traqueurs}
\author{Maxime Catrice}
\date{\today}
\titlegraphic{\includegraphics[height=0.70cm]{logo.png}}

\setbeamertemplate{title page}[default][colsep=-4bp,rounded=true,shadow=false]
\setbeamercolor{background canvas}{bg=backgroundColor}
\setbeamertemplate{blocks}[rounded][shadow=false]
\beamertemplatenavigationsymbolsempty
\newcounter{acolumn}%  Number of current column
\newlength{\acolumnmaxheight}%   Maximum column height
%%%%%%%%%%%%%%%%%%%%%%%%%


\makeatletter

% `column` replacement to measure height
\newenvironment{@acolumn}[1]{%
    \stepcounter{acolumn}%
    \begin{lrbox}{\@tempboxa}%
    \begin{minipage}{#1}%
}{%
    \end{minipage}
    \end{lrbox}
    \@tempdimc=\dimexpr\ht\@tempboxa+\dp\@tempboxa\relax
    % Save height of this column:
    \expandafter\xdef\csname acolumn@height@\roman{acolumn}\endcsname{\the\@tempdimc}%
    % Save maximum height
    \ifdim\@tempdimc>\acolumnmaxheight
        \global\acolumnmaxheight=\@tempdimc
    \fi
}

% `column` wrapper which sets the height beforehand
\newenvironment{@@acolumn}[1]{%
    \stepcounter{acolumn}%
    % The \autoheight macro contains a \vspace macro with the maximum height minus the natural column height
    \edef\autoheight{\noexpand\vspace*{\dimexpr\acolumnmaxheight-\csname acolumn@height@\roman{acolumn}\endcsname\relax}}%
    % Call original `column`:
    \orig@column{#1}%
}{%
    \endorig@column
}

% Save orignal `column` environment away
\let\orig@column\column
\let\endorig@column\endcolumn

% `columns` variant with automatic height adjustment
\NewEnviron{acolumns}[1][]{%
    % Init vars:
    \setcounter{acolumn}{0}%
    \setlength{\acolumnmaxheight}{0pt}%
    \def\autoheight{\vspace*{0pt}}%
    % Set `column` environment to special measuring environment
    \let\column\@acolumn
    \let\endcolumn\end@acolumn
    \BODY% measure heights
    % Reset counter for second processing round
    \setcounter{acolumn}{0}%
    % Set `column` environment to wrapper
    \let\column\@@acolumn
    \let\endcolumn\end@@acolumn
    % Finally process columns now for real
    \begin{columns}[#1]%
        \BODY
    \end{columns}%
}
\makeatother
%%%%%%%%%%%%%%%%%%%%%%%%%

\NewEnviron{frameNoSB}{
\makeatletter
\setbeamertemplate{sidebar canvas left}{}
\setbeamertemplate{sidebar left}{}
\makeatother
\begin{frame}
    \BODY
%\tableofcontents
\end{frame}
}

\NewEnviron{frameTitle}{
\setbeamertemplate{frametitle}[default][center]
\makeatletter
\setbeamertemplate{headline}{\color{backgroundColor}45\newline 45\newline 45\newline 45\newline 45\newline 45\newline 45\newline 45\newline 45\newline 45}
\setbeamertemplate{sidebar canvas left}{}
\setbeamertemplate{sidebar left}{}



\makeatother
\begin{frame}
\begin{minipage}[c]{\linewidth-\beamerleftmargin+\beamerrightmargin}
\BODY
\end{minipage}
\end{frame}
}

\setbeamerfont{title}{series=\bfseries,parent=structure}
\setbeamerfont{subtitle}{size=\Large,series=,parent=structure}

\makeatletter
\newlength\beamerleftmargin
\setlength\beamerleftmargin{\Gm@lmargin}

\newlength\beamerrightmargin
\setlength\beamerrightmargin{\Gm@rmargin}
\makeatother

\newcommand{\nologo}{\setbeamertemplate{logo}{}}

\newcommand{\slidetitle}[1][]{
  \frametitle{\insertsection\ifthenelse{\equal{#1}{}}{}{: #1}}
}
\newcommand{\mono}[1]{
\texttt{#1}
}
%%%%%%%%%%%%%%%%%%%%%%%%%%%%%%%%%%%%%%%%%
%%%%%%%%%%%%%%%%%%%%%%%%%%%%%%%%%%%%%%%%%
%%%%%%%%%%%%%%%%%%%%%%%%%%%%%%%%%%%%%%%%%
%%%%%%%%%%%%%%%%%%%%%%%%%%%%%%%%%%%%%%%%%
%%%%%%%%%%%%%%%%%%%%%%%%%%%%%%%%%%%%%%%%%
%%%%%%%%%%%%%%%%%%%%%%%%%%%%%%%%%%%%%%%%%



  \setbeameroption{show notes on second screen=top}
  \begin{document}

  \section{Légalité et rétroingénierie}
  \begin{frame}[t]
    \slidetitle[]%{Légalité et rétroingénierie}
    \note{\ExecuteMetaData[002-legal_notes.tex]{propIntelect}}
    \note{\onslide+<4->{\ExecuteMetaData[002-legal_notes.tex]{loi}}}
    \note{\onslide+<10->{\ExecuteMetaData[002-legal_notes.tex]{Resume}}}
%    \note{\onslide+<4->{\ExecuteMetaData[002-legal_notes.tex]{interropérabilité}}}
    \begin{columns}
      \begin{column}{0.7\linewidth}
        \begin{block}{Logiciels et propriété intellectuelle}<2->
          \begin{itemize}
          \item<3-> Logiciel protégeable
          \item<4-> Fonctionnalité en tant que telle non protégeable
          \end{itemize}
        \end{block}
        \begin{block}{Article 122-6-1 du code de la propriété intellectuelle}<5->
          \begin{itemize}
          \item<6-> Acquisition légale du logiciel
          \item<7-> Soit:
          \begin{itemize}
          \item<8-> La license ne l'interdit pas
          \item<9-> Réalisation à des fins d'interropérabilité
          \end{itemize} 
          \end{itemize}
        \end{block}
      \end{column}
      \begin{column}{0.25\linewidth}
        \includegraphics[width=0.95\linewidth]{img/justice.png}
      \end{column}
    \end{columns}
      \onslide+<10->{\begin{center}
        « On n'a donc pas le droit en France de démontrer techniquement qu'un logiciel présente des failles
  de sécurité, ou que la publicité pour ces logiciels est mensongère. Dormez tranquilles, citoyens, tous
  vos logiciels sont parfaits. » 
      \end{center}
      \vspace{-1.5\baselineskip}
      \hspace{0.66\linewidth}Guillermito
      }
    \end{frame}


    \begin{frame}
    \slidetitle[Les Bug BugBounty]
    \note{\ExecuteMetaData[002-legal_notes.tex]{BugBounty}}
    \note{\onslide<4->{\ExecuteMetaData[002-legal_notes.tex]{img/Tweet}}}
    \pause
    \begin{block}{Qu'est ce qu'un Bug Bounty?}
      \centering
      \onslide<3->{Un bug bounty est un programme proposé par de nombreux sites web et développeurs de logiciel
       qui permet à des personnes de recevoir reconnaissance et compensation après avoir reporté
        des bugs, surtout ceux concernant des exploits et des vulnérabilités}
    \end{block}
    \vfill
    \centering
    \onslide<4->{
      \begin{figure}
        \includegraphics[width=0.5\linewidth]{tweet.png}
        \caption{Tweet montrant une faille dans macOS}
      \end{figure}}
      \end{frame}
  \end{document}

  Tout d’abord, il faut savoir que si la licence ne dit pas le contraire,
  il n’est pas illicite de s’adonner au reverse au engineering d’un logiciel.
  Si le concepteur d’un logiciel veut s’y opposer, il faut que cela soit
  spécifié dans le fameux contrat que l’on doit normalement lire avant de
  cliquer sur « J’accepte les termes et conditions d’utilisation ».
  Si cela n’est pas mentionné, rien ne s’y oppose.

  Il convient de souligner que si le logiciel peut être protégé par une licence,
  sa fonctionnalité en tant que telle n’est pas protégeable.
  Par exemple, le programme Word inclus dans la suite bureautique
  Office de Windows fait l’objet d’une protection juridique spécifique.
  Mais la fonctionnalité de Word, à savoir être un logiciel de
  traitement de texte, n’est pas protégeable par Microsoft.

  Mais si le soft est interopérable et aisément disponible,
  un utilisateur n’aura pas le droit de le décompiler même pour effectuer
  des maintenances correctives ou évolutives.

  Donc, à partir du moment où l’on se met à trifouiller dans le code source
  d’un logiciel propriétaire pour en comprendre son fonctionnement,
  sans pour autant chercher à le faire tourner sur une autre plateforme
  que celle pour laquelle il a été initialement conçu, est illicite et le
  fait de publier sur Internet les résultats d’une telle expérimentation,
  accessible à un public Français, aggrave les faits.
  « On n'a donc pas le droit en France de démontrer techniquement qu'un logiciel présente des failles
de sécurité, ou que la publicité pour ces logiciels est mensongère. Dormez tranquilles, citoyens, tous
vos logiciels sont parfaits. » Guillermito

En droit français, la simple recherche de failles de sécurité n’est pas réprimée en tant que telle, sous
réserve de respect du régime juridique relatif au respect du droit de la propriété intellectuelle.
C’est la publication des résultats, la « démonstration technique » de ces recherches qui peut être
sanctionnée par l’article 323-3-1 du Code pénal.

BugBounty