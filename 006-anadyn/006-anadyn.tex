\documentclass[aspectratio=1610, handout]{beamer}%, handout
\usepackage{graphics}
\usepackage{pifont}
\usepackage{ulem}
\usepackage{xcolor}
\usepackage{modifycolor}
\usepackage{lipsum}
\usepackage{fontspec}
\usepackage{caption}
\renewcommand{\figurename}{FIGURE}
\captionsetup[figure]{labelfont={color=leftFootlineColor, scriptsize}, textfont={color=normalBlockColor, scriptsize}}
\usepackage{tabularx}
\newcolumntype{Y}{>{\centering\arraybackslash}X}
\usepackage{tikz}
\usepackage{standalone}
\usepackage{svg}
\usepackage{multicol}
\usepackage{catchfilebetweentags}
\usepackage{xifthen}
\usetikzlibrary{arrows}
\usetikzlibrary{backgrounds}
\setmonofont[
  Contextuals={Alternate}
]{FuraCode Nerd Font}
%\usepackage[T1]{fontspec}
%\usepackage[scaled]{berasans}
%\usepackage[scale=1,sfdefault,light]{roboto}
\setsansfont{Roboto Medium}
\usepackage{pgfpages}


\usepackage[cache=false,outputdir=build]{minted}
\definecolor{bg_code}{HTML}{282828}
\usemintedstyle{darcula}
\setminted{
      fontsize=\scriptsize, 
      linenos,
      numbersep=0pt,
      gobble=5,
      framesep=3mm} 
      \renewcommand{\theFancyVerbLine}{\texttt{{{\arabic{FancyVerbLine}}}}}
\usepackage{adjustbox}
\usepackage{environ}% Required for \NewEnviron, i.e. to read the whole body of the environment
%blue/red/pink/purple/yellow/orange/green/gray/bluegray/brown
\usepackage{tikz}
\usetikzlibrary{patterns}
\usepackage[french]{babel}
% This is the file main.tex
\usepackage{beamerthemesideblue}


\setbeamerfont{note page}{size=\footnotesize}
\setbeamertemplate{note page}[custom]
\setbeamercolor{note page}{bg=backgroundColor, fg=white}

%\usecolortheme{red}
\logo{\includegraphics[height=0.70cm]{logo.png}}
\title[]{La rétroingénierie appliquée à Android}
\subtitle{La traque aux traqueurs}
\author{Maxime Catrice}
\date{\today}
\titlegraphic{\includegraphics[height=0.70cm]{logo.png}}

\setbeamertemplate{title page}[default][colsep=-4bp,rounded=true,shadow=false]
\setbeamercolor{background canvas}{bg=backgroundColor}
\setbeamertemplate{blocks}[rounded][shadow=false]
\beamertemplatenavigationsymbolsempty
\newcounter{acolumn}%  Number of current column
\newlength{\acolumnmaxheight}%   Maximum column height
%%%%%%%%%%%%%%%%%%%%%%%%%


\makeatletter

% `column` replacement to measure height
\newenvironment{@acolumn}[1]{%
    \stepcounter{acolumn}%
    \begin{lrbox}{\@tempboxa}%
    \begin{minipage}{#1}%
}{%
    \end{minipage}
    \end{lrbox}
    \@tempdimc=\dimexpr\ht\@tempboxa+\dp\@tempboxa\relax
    % Save height of this column:
    \expandafter\xdef\csname acolumn@height@\roman{acolumn}\endcsname{\the\@tempdimc}%
    % Save maximum height
    \ifdim\@tempdimc>\acolumnmaxheight
        \global\acolumnmaxheight=\@tempdimc
    \fi
}

% `column` wrapper which sets the height beforehand
\newenvironment{@@acolumn}[1]{%
    \stepcounter{acolumn}%
    % The \autoheight macro contains a \vspace macro with the maximum height minus the natural column height
    \edef\autoheight{\noexpand\vspace*{\dimexpr\acolumnmaxheight-\csname acolumn@height@\roman{acolumn}\endcsname\relax}}%
    % Call original `column`:
    \orig@column{#1}%
}{%
    \endorig@column
}

% Save orignal `column` environment away
\let\orig@column\column
\let\endorig@column\endcolumn

% `columns` variant with automatic height adjustment
\NewEnviron{acolumns}[1][]{%
    % Init vars:
    \setcounter{acolumn}{0}%
    \setlength{\acolumnmaxheight}{0pt}%
    \def\autoheight{\vspace*{0pt}}%
    % Set `column` environment to special measuring environment
    \let\column\@acolumn
    \let\endcolumn\end@acolumn
    \BODY% measure heights
    % Reset counter for second processing round
    \setcounter{acolumn}{0}%
    % Set `column` environment to wrapper
    \let\column\@@acolumn
    \let\endcolumn\end@@acolumn
    % Finally process columns now for real
    \begin{columns}[#1]%
        \BODY
    \end{columns}%
}
\makeatother
%%%%%%%%%%%%%%%%%%%%%%%%%

\NewEnviron{frameNoSB}{
\makeatletter
\setbeamertemplate{sidebar canvas left}{}
\setbeamertemplate{sidebar left}{}
\makeatother
\begin{frame}
    \BODY
%\tableofcontents
\end{frame}
}

\NewEnviron{frameTitle}{
\setbeamertemplate{frametitle}[default][center]
\makeatletter
\setbeamertemplate{headline}{\color{backgroundColor}45\newline 45\newline 45\newline 45\newline 45\newline 45\newline 45\newline 45\newline 45\newline 45}
\setbeamertemplate{sidebar canvas left}{}
\setbeamertemplate{sidebar left}{}



\makeatother
\begin{frame}
\begin{minipage}[c]{\linewidth-\beamerleftmargin+\beamerrightmargin}
\BODY
\end{minipage}
\end{frame}
}

\setbeamerfont{title}{series=\bfseries,parent=structure}
\setbeamerfont{subtitle}{size=\Large,series=,parent=structure}

\makeatletter
\newlength\beamerleftmargin
\setlength\beamerleftmargin{\Gm@lmargin}

\newlength\beamerrightmargin
\setlength\beamerrightmargin{\Gm@rmargin}
\makeatother

\newcommand{\nologo}{\setbeamertemplate{logo}{}}

\newcommand{\slidetitle}[1][]{
  \frametitle{\insertsection\ifthenelse{\equal{#1}{}}{}{: #1}}
}
\newcommand{\mono}[1]{
\texttt{#1}
}
%%%%%%%%%%%%%%%%%%%%%%%%%%%%%%%%%%%%%%%%%
%%%%%%%%%%%%%%%%%%%%%%%%%%%%%%%%%%%%%%%%%
%%%%%%%%%%%%%%%%%%%%%%%%%%%%%%%%%%%%%%%%%
%%%%%%%%%%%%%%%%%%%%%%%%%%%%%%%%%%%%%%%%%
%%%%%%%%%%%%%%%%%%%%%%%%%%%%%%%%%%%%%%%%%
%%%%%%%%%%%%%%%%%%%%%%%%%%%%%%%%%%%%%%%%%



\usepackage{xmpmulti}
 \setbeameroption{show notes on second screen=top}
\begin{document}
 % Smali: langage lisible par l'humain des fichiers dex

 \section{L'analyse dynamique}
\begin{frame}
  \slidetitle[]
  \note{\onslide+<1->{\ExecuteMetaData[006-anadyn_notes.tex]{Interet}}}
  \note{\onslide+<7->{\ExecuteMetaData[006-anadyn_notes.tex]{Environnement}}}
  \note{\onslide+<12->{\ExecuteMetaData[006-anadyn_notes.tex]{Utilisation}}}
  \begin{columns}
    \begin{column}{0.48\linewidth}
      \begin{block}{Intérêt:}<2->
        \begin{itemize}
        \item<3-> Obtenir des informations générées dynamiquement par l'application
        \item<4-> Difficulté de déchiffre des strings lourdement obfusqués
        \item<5-> Requêtes qui ne peuvent pas être interprétées par un MITM
        \end{itemize}
      \end{block}
    \end{column}
    \begin{column}{0.48\linewidth}
      \begin{block}{Environnemnt utilisé:}<6->
        \begin{itemize}
        \item<7-> \uline{Émulateur:} Genymotion
        \item<8-> Root, Xposed
        \item<9-> Inspeckage
        \item<10-> Android Device Monitor
        \end{itemize}
      \end{block}
      \begin{block}{Utilisation:}<11->
        \begin{itemize}
        \item<12-> Utilisation d'un débugger
        \item<13-> Analyse de la mémoire utilisée par l'application
      \end{itemize}
      \end{block}
    \end{column}
  \end{columns}
\end{frame}
\begin{frame}
  \slidetitle[Debugger]
  \note{\onslide+<1->{\ExecuteMetaData[006-anadyn_notes.tex]{Debugger}}}
  \begin{columns}
    \begin{column}{0.7\linewidth}
      \begin{block}{Principe}<2->
        \begin{enumerate}
        \item<3-> Décompilation de l'application
        \item<4-> Import du projet dans Android Studio
        \item<5-> Mise en place des points d'arrêts
        \item<6-> Lancement du mode debug
        \item<7-> Analyse de l'état de l'application aux points d'arrêts 
        \end{enumerate}
        \begin{itemize}
        \item<8-> Il est par la suite possible de recompiler l'application avec les modifications apportés au smali
        \end{itemize}
      \end{block}
    \end{column}
    \begin{column}{0.25\linewidth}     
      \centering
     \begin{overlayarea}{0.6\linewidth}{\linewidth}
      \only<1-2>{\includegraphics[width=\linewidth]{0_bugdroid-0.png}}
      \only<3-3|handout:0>{\includegraphics[width=\linewidth]{img/0_bugdroid-1.png}}
      \only<4-4|handout:0>{\includegraphics[width=\linewidth]{img/0_bugdroid-2.png}}
      \only<5-5|handout:0>{\includegraphics[width=\linewidth]{img/0_bugdroid-3.png}}
      \only<6-6|handout:0>{\includegraphics[width=\linewidth]{img/0_bugdroid-4.png}}
      \only<7-7|handout:0>{\includegraphics[width=\linewidth]{img/0_bugdroid-5.png}}
      \only<8-8|handout:0>{\includegraphics[width=\linewidth]{img/0_bugdroid-6.png}}
     \end{overlayarea}
    \end{column}
  \end{columns}

\end{frame}
\end{document}