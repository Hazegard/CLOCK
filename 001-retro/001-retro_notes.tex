%<*Definition>
\textbf{\uline{Rétro-ingénierie:}}\\
\textbullet~~Analyser un programme sans ses sources pour en comprendre le fonctionnement interne.
\\\textbullet~~Deux approches complémentaires:
\\~~--\uline{Analyse statique}: \\
Reconstituer le code source du logiciel à partir d'un exécutable ou au moins de le traduire dans le langage assembleur.
\\~~--\uline{Analyse dynamique}.\\
Étudier le programme directement pendant son exécution à l'aide d'un débogueur. 
%</Definition>

%<*Objectifs>
\bigbreak
\textbf{\uline{Objectifs:}}
\smallbreak
\textbf{\uline{Interopérabilité d'un logiciel:}}\\ afin d’en comprendre le fonctionnement et ainsi le rendre compatible avec d'autres logiciels
\smallbreak
\textbf{\uline{Documentation:}}\\ Retrouver le fonctionnement d’un logiciel avec lequelle on souhaiterait communiquer, mais dont la documentation n’est plus disponible. 
\smallbreak
\textbf{\uline{Veille compétitive:}}\\ Étudier les produits concurrents, les méthodes utilisées, déceler d’éventuelles violations de brevet par un concurrent. 
\smallbreak
\textbf{\uline{Recherche de failles de sécurité:}}\\ Failles de sécurités dans les applications commerciales dont les sources ne sont pas disponibles. 
Les Virus sont eux aussi systématiquement étudiés par rétro ingénierie.
\smallbreak
\textbf{\uline{Piratage:}}\\ Prolonger la période d'essaie. `amtlib.dll' par exemple (adobe). 
%</Objectifs>