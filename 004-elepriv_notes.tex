%<*Elepriv>
L'élévation de privilège résulte de l'octroi à un intrus d'autorisations 
supérieures à celles initialement accordées.Par exemple, un intrus avec 
un jeu de privilèges contenant des autorisations « en lecture seule »
élèvent d'une façon ou d'une autre le jeu pour inclure des autorisations
« en lecture et en écriture ».
%</Elepriv>

%<*Root>
\textbf{\uline{Root:}}\\
Utilisation de privilèges avances, permettant de limiter des limitations 
imposées par le système\\
Par exemple, cela permet de supprimer les applications systèmes,
qui ne sont pas désinstallables en tant que simple utilisateur.\\
%</Root>

%<*RootPrincipe>
\textbf{\uline{Principe du root:}}\\
Utilisation du'une faille d'android, ou alors du mode récupération d'android pour
obtenir temporairement un uid à , c'est à dire root\\
Remontage de la partition système en écriture, afin de pouvoir la modifier\\
Copie de nouveaux binaires, tels que su, busybox\\
Remontage de la partition système en lecture seuls
%</RootPrincipe>

%<*RootExemples>
\textbf{\uline{Exemples d'utilisation:}}\\
\textbullet~~Accéder aux partitions systèmes\\
\textbullet~~Installation de busybox\\
\textbullet~~Sauvegarder une application en conservant l'état de l'application
au moment de la sauvegarde\\
\textbullet~~Modifier des propriétés systèmes (densité d'écran, adresse mac...)\\
%</RootExemples>
%<*Xposed>
\textbf{\uline{Xposed:}}\\
Framework permettant d’intercepter toutes méthodes d’une application, pour injecter
du code suplémentaireExemple d’utilisation\\
Ne fonctionne qu'avec les applications java, mais pas avec les bibrairies natives, par exemple\\
%</Xposed>

%<*XposedUtilisation>
\textbf{\uline{Exemples d'utilisation de Xposed:}}\\
\textbullet~~Lire les paramètres des applications\\
\textbullet~~Désactiver la vérification SSL, pour par exemple, pouvoir déchiffre le traffic\\
\textbullet~~Modifier son IMEI\\
\textbullet~~Simuler sa position GPS\\
%</XposedUtilisation>

%<*BootAndroid>
\textbf{\uline{Le démarrage d'Android:}}\\
\textbullet~~On s'intéresse au processus une fois le lancement du kernel\\
\textbullet~~Le kernel initialise le processus Init\\
\textbullet~~Init lance à son tour des démons (usb, adb, ril), et le runtime\\
\textbullet~~Init lance aussi Zygote\\
%</BootAndroid>

%<*Zygote>
\textbf{\uline{Zygote:}}\\
Zygote est un processus primordiale pour Android:\\
\textbullet~~Il initialise la machine virtuelle\\
\textbullet~~pré-charges des classes communes aux applications\\
\textbullet~~Se fork pour chaque nouvelle application lancée\\
\textbullet~~Partage une partie de sa mémoire avec les applications forkées\\
%</Zygote>

%<*XposedFonctionnement>
\textbf{\uline{Fonctionnement de Xposed:}}\\
\textbullet~~Le processus init est modifié pour changer le comportement de Zygote en ajoutant
des librairies au classpath\\
\textbullet~~Ajout de librairies à Zygote permettant de détecter le lancement d'application\\
\textbullet~~A chaque lancement d'une application, Zygote va remplacer le code de l'application
pour injecter du code externe\\
%</XposedFonctionnement>