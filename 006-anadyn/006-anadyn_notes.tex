%<*Interet>
\textbf{\uline{Intérêt:}}\\
\textbullet~~Informations générées dynamiquement ne peuvent être récupérée par 
 une analyse statique\\
\textbullet~~Certaines méthodes d'obfuscation sont difficiles à déchiffrer.~
Cependant, étant donné que ces valeurs vont être 
  déchiffrées au cours de l'execution, on peut essayer de la récupérer à ce moment\\
\textbullet~~ Certaines requêtes ne peuvent être déchiffrée par un MITM,
 on peut alors essayer de lire les données de la requête au moment de l'envoi~
 ou de la réception de la requête
%</Interet>

%<*Utilisation>
\bigbreak
\textbf{\uline{Utilisation:}}\\
\uline{Débugger:}Permet de faire mettre des breakpoints~
Cependant, étant donné qu'on a pas accès au code source,~
 il est nécessaire de décompiler l'application en smali,~
 reconstruire le projet et recompiler l'application\\
\uline{Mémoire:} Permet de récupérer certaines valeurs 
%</Utilisation>

%<*Environnement>
\bigbreak
\uline{Environnement:}\\
\uline{Emulateur:} plus de facilité pour le rooter ainsi qu'installer Xposed\\
\uline{Root, Xposed:}\\
\uline{Inspeckage:}Démarrer des activités non déclarées, Désactiver le SSL, remplacer des paramètres d'application...\\
\uline{Android Device Monitor:} Outil intégré à Android Studio~
 offrant des fonctions de débug et d'analyse d'application
%</Environnement>

%<*Debugger>
\bigbreak
\textbf{\uline{Principe}}:\\
\textbullet~~ On décompile l'APK, mais on reste au stade du smali\\
\textbullet~~On importe les fichiers dans Android Studio pour y générer un nouveau projet\\
\textbullet~~On place les points d'arrêts\\
\textbullet~~On lance l'application\\
\textbullet~~On analyse l'état de la mémoire de l'application aux points d'arrêts\\

Enfin, si on souhaite produire une version modifiée de l'application, il est possible de recompiler
le smali pour produire un nouveau APK
%</Debugger>